\section{Installation}
\label{sec:installation}

For the purpose of your project, I strongly encourage you to use
OverLeaf~\cite{OverleafOnlineLaTeX} while you become familiar with this new and exciting
technology. Once you feel a desire to learn more about \LaTeX, such as in the summer after your
project of choice, it is worth considering writing something in a code editor or \acrshort{ide}
such as \acrfull{vscode}, as I am for this template.

\subsection{OverLeaf}
\label{sec:installation/OverLeaf}

OverLeaf (\url{www.overleaf.com}) is an online \LaTeX\ editor that makes it easy to get started
with using \LaTeX\, avoiding having to install a TeX distribution such as TeX
Live~\cite{TeXLiveTeX} onto your PC and letting you share your writeup with your supervisor to
make getting feedback as easy as possible.

As a disclaimer, these steps may not be perfectly accurate, since I'm writing them from memory and
they are, of course, subject to change.

\begin{enumerate}
	\item Navigate to \url{www.overleaf.com} and create an account using your Oxford login. Either
	      your college (e.g., your.name@college.ox.ac.uk) or your department (e.g.,
	      your.name@cs.ox.ac.uk) alias is fine, since you can later add the other as a secondary
	      email. By using an Oxford email, you will get free access to the premium features.
	\item In your user settings, link your OverLeaf account to a GitHub account.
	      \todo[inline]{Expand on this. I don't remember how this goes}
	\item Go to \url{github.com/Ned-Stevenson/OxCSProject} and create a fork of OxCSProject. This
	      will create a new repo that you own as a copy of OxCSProject, which you can then import
	      into OverLeaf.
	\item At the OverLeaf home page (\url{www.overleaf.com/project}), create a new project, and
	      select "Import from GitHub". From the list, select your forked repo.
\end{enumerate}

\subsection{Offline}
\label{sec:installation/offline}

To install offline, you need only clone this repository into whichever folder you'd like to work
in. I would recommend making a fork of the repo and cloning that so that you can version control
your work, protecting your efforts in the case of loss, damage, or theft of your favourite \LaTeX\
writing device. You will also need to install something to render your \LaTeX\ into a PDF, such as
TeX Live~\cite{TeXLiveTeX} and an \acrshort{ide} to work in.

\begin{enumerate}
	\item (Optionally) Go to \url{github.com/Ned-Stevenson/OxCSProject} and create a fork of
	      OxCSProject
	\item Clone this repo or your fork onto your local device:

	      \verb*|git clone https://github.com/Ned-Stevenson/OxCSProject.git|
	\item Install a tool such as TeX Live~\cite{TeXLiveTeX}, following any relevant instructions.
\end{enumerate}

I can also recommend the LaTeX Workshop \acrshort{vscode} extension~\cite{LaTeXWorkshopVisual} as
well as latexindent~\cite{hughesCmhughesLatexindentpl2024} to make it easy to both render your
code as well as keep it tidy and readable. Note that latexindent comes pre-installed with TeX
Live, and the configuration options that I used to format this project are available in the GitHub
repository.
